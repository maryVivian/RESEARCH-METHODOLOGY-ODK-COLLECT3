\documentclass{article}
\usepackage{graphicx}
\begin{document}
\title{A REPORT ABOUT THE BUILDINGS IN MAKERERE UNIVERSITY.}
\author{NANTALE MARY VIVIAN }
\maketitle
\tableofcontents
\section{EXECUTIVE SUMMARY }{The research report is about some of the existing buildings forming the great infrastructure of Makerere University. Here I was able to find out the positions of the buildings using GPS coordinates, the pictures of the buildings, their names, purpose in the university, maximum number of people the different buildings can accommodate, and number of floors for storied buildings. This would help new people in the university to find different places with ease.
\section{INTRODUCTION}{ Makerere University covers a very wide area and consists of a variety of infrastructure. This makes it hard for new people to the university to get access to the various buildings. Therefore, research has been carried out to get such information so as to solve this problem.}
 
\section{METHODOLOGY}{I used the method of observation to see the different buildings around Makerere University. I gathered the information using a mobile application, ODK Collect. Here, i was able to take pictures of the different buidings and find their location using GPS coordinates.}

\section{The description of various buildings. }{The different buildings in Makerere University consist of students’ halls of residence, faculty buildings, library book banks, places of worship like a mosque, administration blocks, and restaurants. These all serve different purposes to the students, teaching and non-teaching staff. The buildings are either constructed with blocks or other materials like wood.}

\section{Problems faced when collecting data. }{\itemize

\item Unforeseen weather changes where the rain was a hindrance to my movement around the different buildings.	

\item Poor connection within the university which couldn’t enable me connect to the server efficiently.

\item High security measures on some buildings which would make it illegal to take pictures because it would be considered as a threat.

\item Executing machine code is faster than interpreting byte code 

\item Lack of accuracy when finding the GPS coordinates.
}
\section{Possible solutions to the problems. }{\itemize

\item Better means of transportation around campus as kit would make the research work faster.

\item Communication towers are being built in Makerere University so as to deal with the problem of poor network.


\section{Conclusion}{
In conclusion, this research will help to solve the problem of searching for the different buildings depending on one’s needs in the University.
}
\section {Sample Screenshots of the Phone Application.} 
\begin{figure}[!htb]

\minipage{0.24\textwidth}
\includegraphics[width=\linewidth,scale=0.5]{Screenshot1.png}
\caption{Screenshot 1}
\endminipage\hfill
\minipage{0.24\textwidth}
\includegraphics[width=\linewidth,scale=0.5]{Screenshot2.png}
\caption{Screenshot 2}
\endminipage\hfill
\minipage{0.24\textwidth}
\includegraphics[width=\linewidth,scale=0.5]{Screenshot3.png}
\caption{Screenshot 3}
\endminipage\hfill
\minipage{0.24\textwidth}
\includegraphics[width=\linewidth,scale=0.5]{Screenshot4.png}
\caption{Screenshot 4}
\endminipage\hfill



\end{figure}
\begin{figure}
\includegraphics[width=\linewidth,scale=0.5]{Capture.png}
\caption{Capture}
\end{figure}


\end{document}